\section{Введение в теорию случайных процессов-2}
Сделаю небольшое лирическое отступление и вспомним гауссовские векторы. Для 
этого вспомним, что такое \emph{характеристическая функция} случайной величины 
и вектора.
\begin{definition}
    Пусть \(\xi\)~--- случайная величина с плотностью \(p_{\xi}\). Тогда её 
    характеристической функцией называется функция \(\phi_{\xi} : \R \mapsto 
    \C\), определяемая следующим образом:
    \[
        \phi_{\xi}(t) = \E{e^{it\xi}} = \int\limits_{-\infty}^{+\infty} 
        e^{itx}p_{\xi}(x)\diff x.
    \]
\end{definition}
\begin{definition}
    Пусть \(\bm{\xi} = (\xi_{1}, \dots, \xi_{n})\)~--- случайный вектор с 
    совместной плотностью \(p_{\bm{\xi}}\). Тогда её характеристической 
    функцией называется функция \(\phi_{\bm{\xi}} : \R^{n} \mapsto \C\), 
    определяемая следующим образом: \(\phi_{\bm{\xi}}(\mathbf{t}) = 
    \E{e^{i\langle\bm{\xi}, \mathbf{t}\rangle}}\), где \(\langle\cdot, 
    \cdot\rangle\)~--- скалярное произведение в \(\R^{n}\).
\end{definition}

По сути, характеристическая функция~--- это преобразование Фурье функции 
распределения.

Далее, из курса теории вероятности известно, что если \(\xi \sim 
\mathcal{N}(\mu, \sigma^{2})\), то
\[
    \phi_{\xi}(t) = \exp\left\{i \mu t - \frac{1}{2}\sigma^{2}t^{2}\right\}.
\]

Поэтому гауссовский вектор вводят следующим образом:
\begin{definition}
    Случайный вектор \(\bm{\xi} = (\xi_{1}, \dots, \xi_{n})\) подчиняется 
    \emph{многомерному нормальному распределению}, если его характеристическая 
    функция равна
    \[
        \phi_{\bm{\xi}}(\mathbf{t}) = \exp\left\{i\langle\bm{\mu}, 
        \mathbf{t}\rangle - \frac{1}{2}\langle\mathbf{\Sigma t}, 
        \mathbf{t}\rangle\right\},
    \]
    где \(\bm{\mu} \in \R^{n}\)~--- некоторый фиксированный вектор, а 
    \(\mathbf{\Sigma}\)~--- некоторая симметрическая и неотрицательно 
    определённая матрица. В таком случае пишут, что \(\xi \sim 
    \mathcal{N}(\bm{\mu}, \mathbf{\Sigma})\).
\end{definition}