\section{Семинар от 24.01.2017}
Разберём задачи, заданные на дом.
\begin{problem}
	Найдите характеристическую функцию случайной величины \(\xi\), если она 
	имеет
	\begin{enumerate}[label=(\alph*)]
		\item геометрическое распределение с параметром \(p\);
		\item равномерное распределение на отрезке \([a, b]\);
		\item гамма-распределение \(\Gamma(n, 1)\).
	\end{enumerate}
\end{problem}
\begin{proof}[Решение]
	Вообще, эта задача на применение определения характеристической функции. Из 
	него следует, что
	\[
	\phi_{\xi}(t) = \E{e^{it\xi}} = \int\limits_{-\infty}^{+\infty} 
	e^{itx}p_{\xi}(x)\diff x
	\]
	Однако это верно только для случайных величин, у которых есть плотность. 
	Если распределение дискретно, то нужно считать немного по-другому. Пусть 
	\(X\)~--- это множество значений \(\xi\). Тогда
	\[
	\phi_{\xi}(t) = \E{e^{it\xi}} = \sum_{x \in X} e^{itx}\Pr{\xi = x}
	\]
	Теперь помжно приступать к самому решению.
	\begin{enumerate}[label=(\alph*)]
		\item Напомню, что для геометрического распределения \(X = \Z_+\) и 
		\(\Pr{\xi = n} = p(1 - p)^n\). Тогда
		\[
		\phi_{\xi}(t) = \sum_{n = 0}^{\infty} e^{itn}p(1 - p)^n = p\sum_{n = 
		0}^{\infty} \left(e^{it}(1 - p)\right)^n = \frac{p}{1 - (1 - p)e^{it}}
		\]
		\item Плотность \(U(a, b)\) равна \(\frac{1}{b - a}\I\{x \in [a, 
		b]\}\). Тогда
		\[
		\phi_{\xi}(t) = \frac{1}{b - a}\int\limits_{a}^{b} e^{itx}\diff x = 
		\frac{e^{itb} - e^{ita}}{(b - a)it}
		\]
		\item Плотность \(\Gamma(n, 1)\) равна
		\[
		p_{\xi}(x) = \frac{x^{n - 1}e^{-x}}{(n - 1)!}\I\{x \geq 0\}
		\]
		Тогда
		\[
		\phi_{\xi}(t) = \frac{1}{(n - 1)!}\int\limits_{0}^{+\infty} e^{itx}x^{n 
		- 1}e^{-x}\diff x
		\]
		Возникает не самый приятный интеграл. Конечно, можно взять его по 
		частям \(n\) раз, но не захочется. К сачстью, есть более простой путь.
		Вспомним, что
		\[
		\left(\phi_{\xi}(t)\right)^{(n)} = \E{(i\xi)^ne^{it\xi}}
		\]
		Далее, введём функцию
		\[
		f(t) = \int\limits_{0}^{+\infty} e^{itx}e^{-x}\diff x = \frac{1}{1 - it}
		\]
		Осталось заметить, что \(f(t)\) тоже является характеристической 
		функцией (для экспоненциального стандартного распределения) и
		\[
		\phi_{\xi}(t) = \frac{f^{(n - 1)}(t)}{i^{n - 1}(n - 1)!} = \frac{i^{n - 
		1}(-1)(-2)(-3)\dots(-(n - 1))}{(1 - it)^{n}i^{n - 1}(n - 1)!} = (1 - 
		it)^{-n}\qedhere
		\]
	\end{enumerate}
\end{proof}

\begin{problem}
	Пусть \(\phi_{\xi}(t)\)~--- характеристическая функция случайной величины 
	\(\xi\). Являются ли характеристическими функции: 1) \(\Re \phi_{\xi}(t)\), 
	2) \(\Im\phi_{\xi}(t)\), 3) \(|\phi_{\xi}(t)|^2\)?
\end{problem}
\begin{proof}[Решение]
	Для проверки того, является ли функция характеристической, можно привести 
	случайную величину, которая обладает именно такой характеристической 
	функцией. Если же нужно доказать обратное, то нужно показать, какое из 
	свойств ломается.
	
	Будем решать этот номер в порядке \(2 \to 3 \to 1\).
	\begin{enumerate}
		\item Покажем, что \(\Im\phi_{\xi}(t)\) не может быть 
		характеристической функцией. Для этого вспомним, что \(\phi_{\xi}(0) = 
		1\). Отсюда получаем, что \(\Im\phi_{\xi}(0) = 0\). А по свойству 
		характеристических функций \(\Im\phi_{\xi}(0)\) должно быть равно 1. 
		Противоречие.
		\item Вспомним, что
		\[
		|\phi_{\xi}(t)|^2 = \phi_{\xi}(t)\overline{\phi_{\xi}(t)} = 
		\phi_{\xi}(t)\phi_{-\xi}(t)
		\]
		Пусть \(\eta\)~--- случайная величина, независимая с \(\xi\) и \(\eta 
		\eqdist \xi\). Тогда \(\xi - \eta\) будет иметь именно такую 
		характеристическую функцию.
		\item Теперь покажем, что \(\Re \phi_{\xi}(t)\) тоже является 
		характеристической функцией. Пусть \(\eta\)~--- случайная величина, 
		независимая с \(\xi\) и она равновероятно принимает значения из \(\{-1, 
		1\}\). Тогда
		\[
		\phi_{\xi\eta}(t) = \E{e^{it\xi\eta}} = \frac{1}{2}\E{e^{it\xi}} + 
		\frac{1}{2}\E{e^{-it\xi}} = \frac{\phi_{\xi}(t) + 
		\overline{\phi_{\xi}(t)}}{2} = \Re \phi_{\xi}(t).\qedhere
		\]
	\end{enumerate}
\end{proof}

\begin{problem}
	Пусть \(\phi_{\xi}(t)\)~--- характеристическая функция. Покажите, что 
	выполняются неравенства
	\begin{enumerate}[label=(\alph*)]
		\item \(1 - \Re \phi_{\xi}(2t) \leq 4(1 - \Re \phi_{\xi}(t))\)
		\item \((\Im \phi_{\xi}(t))^2 \leq \frac{1}{2}(1 - \Re\phi_{\xi}(2t))\)
		\item \((\Re \phi_{\xi}(t))^2 \leq \frac{1}{2}(1 + \Re\phi_{\xi}(2t))\)
	\end{enumerate}
\end{problem}
\begin{proof}[Решение]
	Перед тем, как решать эту задачу, разберёмся с тем, как вообще устроены 
	\(\Re \phi_{\xi}\) и \(\Im \phi_{\xi}\). Для этого воспользуемся формулой 
	Эйлера и линейностью матожидания:
	\[
	\phi_{\xi}(t) = \E{e^{it\xi}} = \E{\,\cos(t\xi) + i\sin(t\xi)} = 
	\E{\,\cos(t\xi)} + i\,\E{\,\sin(t\xi)}
	\]
	Отсюда видно, что \(\Re \phi_{\xi}(t) = \E{\,\cos(t\xi)}\) и \(\Im 
	\phi_{\xi}(t) = \E{\,\sin(t\xi)}\).
	Теперь посмортим на сами неравенства:
	\begin{enumerate}[label=(\alph*)]
		\item Распишем обе части неравенства:
		\begin{align*}
			1 - \Re \phi_{\xi}(2t) &= 1 - \E{\,\cos(2t\xi)} = \E{1 - 
			\cos(2t\xi)} = \E{2\sin^2(t\xi)} \\
			&= \E{2(1 - \cos(t\xi)(1 + \cos(t\xi))} \\
			4(1 - \Re \phi_{\xi}(t)) &= \E{4(1 - \cos(t\xi)}
		\end{align*}
		Теперь заметим, что \(2(1 - \cos(t\xi)(1 + \cos(t\xi)) \leq 4(1 - 
		\cos(t\xi)\) и вспомним следующий факт: если \(\xi \leq \eta\), то 
		\(\E{\xi} \leq \E{\eta}\). Отсюда сразу получаем желаемое.
		\item Провернём то же самое:
		\begin{align*}
			(\Im \phi_{\xi}(t))^2 &= \left(\E{\,\sin{t\xi}}\right)^2 \\
			\frac{1}{2}(1 - \Re\phi_{\xi}(2t)) &= \E{\frac{1 - \cos(2t\xi)}{2}} 
			= \E{\,\sin^2(t\xi)}
		\end{align*}
		Теперь вспомним, что дисперсия случайной величины неотрицательна. Тогда
		\[
		\D{\,\sin(t\xi)} = \E{\,\sin^2(t\xi)} - \left(\E{\,\sin{t\xi}}\right)^2 
		\geq 0
		\]
		А это и означает, что \((\Im \phi_{\xi}(t))^2 \leq \frac{1}{2}(1 - 
		\Re\phi_{\xi}(2t))\).
		\item Этот пункт абсолютно аналогичен пункту (b).
	\end{enumerate}
\end{proof}

\begin{problem}
	При каких неотрицательных целых \(n\) функция \(\phi_{\xi}(t) = 
	e^{-|t|^n}\) является характеристической?
\end{problem}
\begin{proof}
	Для начала заметим, что при \(n = 0\) это не характеристическая функция, 
	ибо в нуле не 1, а при \(n = 1\) и \(n = 2\) \(\phi_{\xi}(t)\)~--- 
	характеристическая функция. Действительно, \(e^{-|t|}\) соответствует 
	стандартному распределению Коши, а \(e^{-t^2}\) соответствует 
	\(\mathcal{N}(0, 2)\).
	
	Докажем, что при \(n \geq 3\) \(\phi_{\xi}(t)\) уже не является 
	характеристической функцией. Вспомним про неравенство \(1 - \Re 
	\phi_{\xi}(2t) \leq 4(1 - \Re \phi_{\xi}(t))\), которое должно выполняться 
	для всех \(t\). Тогда при \(t \to 0\)
	\[
	1 - e^{-2^n|t|^n} \leq 4(1 - e^{-|t|^n}) \implies 2^n|t|^n + o(t^n) \leq 
	4|t|^n + o(t^n)
	\]
	Отсюда получаем, что \(n \leq 2 + o(1)\) при \(t \to 0\). Следовательно, 
	\(n \leq 2\). 
\end{proof}

Теперь можно перейти к центральной предельной теореме и задачам на её 
применение. На всякий случай напомню формулировку:
\begin{theorem}[Центральная предельная]
	Пусть \(\{\xi_n, n \in \N\}\)~--- последовательность независимых и 
	одинаково распределённых случайных величин, для которых \(0 < \D{\xi_i} < 
	+\infty\). Далее, положим \(S_n = \xi_1 + \dots + \xi_n\). Тогда
	\[
	\frac{S_n - \E{S_n}}{\sqrt{\D{S_n}}} \xrightarrow{d} \mathcal{N}(0, 1).
	\]
\end{theorem}
По-другому это можно записать следующим образом:
\[
\forall x \in \R\ \Pr{\frac{S_n - \E{S_n}}{\sqrt{\D{S_n}}} \leq x} 
\xrightarrow[n \to \infty]{} \Phi(x) = \int\limits_{-\infty}^{x} 
\frac{1}{\sqrt{2\pi}}e^{-\frac{y^2}{2}}\diff y
\]
\begin{problem}
	Пусть есть последовательность независимых случайных величин \(\{\xi_n \mid 
	n \in \N\}\), в которой \(\xi_i \sim \mathrm{Bern}(p)\) (\(p \in [0, 
	1]\)~--- константа). Предположим, что \(p\) неизвестно, но у нас есть \(n\) 
	известных наблюдений \(\xi_{1}, \dots, \xi_{n}\). Как оценить \(p\)?
\end{problem}
\begin{proof}[Решение]
	Опять же, введём обозначение \(S_n = \xi_{1} + \dots + \xi_{n}\). Тогда 
	\(\E{S_n} = np\), а \(\D{S_n} = np(1 - p)\). Тогда по ЦПТ:
	\[
	\Pr{\left|\frac{S_n - np}{\sqrt{np(1 - p)}}\right| \leq x} \approx \Phi(x) 
	- \Phi(-x).
	\]
	Хотелось бы сказать, что наша оценка на \(p\) выполнена почти наверное. 
	Тогда скажем, что \(\Phi(x) - \Phi(-x) \geq 0.99\). Можно проверить, что 
	значение \(x \approx 2.58\) вполне подходит. Тогда
	\[
	\Pr{\left|\frac{S_n}{n} - p\right| \leq \frac{2.58\sqrt{p(1 
	-p)}}{\sqrt{n}}} \approx 0.99
	\]
	Заметим, что \(p(1 - p) \leq \frac{1}{4}\). Поэтому
	\[
	\Pr{\left|\frac{S_n}{n} - p\right| \leq \frac{2.58}{2\sqrt{n}}} \geq 0.99
	\]
	Отсюда получаем, что
	\[
	p \in \left[\frac{S_n}{n} - \frac{1.39}{\sqrt{n}}, \frac{S_n}{n} + 
	\frac{1.39}{\sqrt{n}}\right] \text{ с вероятностью не менее } 0.99.\qedhere
	\]
\end{proof}
На лекции была задача об оценке параметров нормального распределения по 
известным наблюдениям. Для того, чтобы решить её, нужны две теоремы~--- теорема 
о непрерывной сходимости и лемма Слуцкого.
\begin{theorem}[О наследовании сходимости]
	Пусть \(\{\xi_{n} \mid n \in \N\}, \xi\)~--- случайные величины и 
	\(h(x)\)~--- функция, непрерывная относительно распределения \(\xi\) (то 
	есть \(\exists B \in \B(\R)\) такое, что \(h(x)\) непрерывна на \(B\) и 
	\(\Pr{\xi \in B} = 1\)).
\end{theorem}

\[
\phi_{\xi\eta}(t) = \E{e^{it\xi\eta}} = \iint\limits_{\R^2} 
e^{itxy}p_{\xi}(x)p_{\eta}(y)\diff x \diff y \stackrel{?}{=} 
\int\limits_{-\infty}^{+\infty} p_{\xi}(x)dx\int\limits_{-\infty}^{+\infty} 
e^{itxy}p_{\eta}(y)dy
\]